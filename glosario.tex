%http://www.latex-community.org/index.php?option=com_content&view=article&id=263:glossaries-nomenclature-lists-of-symbols-and-acronyms&catid=55:latex-general&Itemid=114
\newglossaryentry{driver}{
	name=driver,
	description={
		Es un traductor que permite al sistema operativo comunicarse con un dispositivo de hardware y viceversa \cite{wheeldin2005no}
	}
}


\newglossaryentry{middleware}{
	name=middleware,
	description={
		Es un modulo intermedio que actuá como conductor entre dos módulos de software \cite{cuerpo}
	}
}

\newglossaryentry{multiplataforma}{
	name=multiplataforma,
	description={
		Perteneciente a entornos informáticos heterogéneos. Por ejemplo, una aplicación de plataforma cruzada es uno que tiene un único código base para múltiples sistemas operativos.	\nocite{Desing}
	}
}


\newglossaryentry{framework}{
	name=framework,
	description={
		Un framework es un conjunto de bibliotecas, herramientas y normas a seguir qye ayudan a desarrollar aplicaciones \cite{Expert}
	}
}

\newglossaryentry{EMMA}{
	name={EMMA},
	description={
		Herramienta usada para representar información extraida automaticamente de la entrada de un usuario por un componente interprete, dicha entrada es tomada en cualquier modalidad soportada por la plataforma. Los componentes que niegan el marcado de EMMA son: reconocimiento de habla, reconocimiento de escritura, ingenierías de entendimiento de lenguaje natural, otros medios interpretes de entrada (como teclado o joystick) y componentes de integración multimodal. \cite{EMMA}
	}
}